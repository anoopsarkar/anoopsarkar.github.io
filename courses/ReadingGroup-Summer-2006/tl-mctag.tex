
\documentclass{article}

\usepackage{txfonts}
\usepackage{color}
\usepackage{latexsym}
\usepackage{strikeout}
\ifx\pdfoutput\undefined
  \usepackage{graphicx}
\else
  \usepackage[pdftex]{graphicx}
  \DeclareGraphicsRule{*}{mps}{*}{}
\fi

\usepackage{amssymb}

\begin{document}

Consider the synchronous tree-local multicomponent TAG (syncMCTAG) shown below. I'll call the left component of the synchronous pair the source, and the right component the target. 

\[ \beta: \left\{ \ 
\begin{array}{c}
\includegraphics{figures.mps.35}
\end{array}
\ , \ 
\left( 
\begin{array}{ccc}
\includegraphics{figures.mps.36} & , &  
\includegraphics{figures.mps.37} 
\end{array}
\right) \ 
\right\} \]

\[ \alpha: \left\{ \ 
\begin{array}{c}
\includegraphics{figures.mps.38}
\end{array}
\ , \ 
\begin{array}{c}
\includegraphics{figures.mps.39} \ \ 
\end{array}
\ 
\right\} \]

The subscripted number $1$ in the tree is a link, and since source TAG in $\alpha$ links to target TAG in $\alpha$ that is going to have a tree-set adjoin into it, we have to denote this in some way. The notation used here to denote this is that link $1$ from source links to $1.1$ in the target which takes the first component and $1.2$ takes the second component of the tree-set. 

In this notation it is not clear to me what happens when an adjunction into $\beta$ does not have a tree-set on the target side. It could pick $1.1$ in the target tree by default.

An example derivation in this syncMCTAG would be 
\[ \alpha(\beta_{[0,(0,00)]}(\beta_{[01,(0,01)]})) \]
which derives the string pair $(aabbccdd, aabb)$. The subscripts indicate the Gorn address in the parent where the adjunction occurs.

The tree-set for syncMCTAG is related to adding a pair of obligatory adjunction constraints on the elementary tree. Does the embedded tree transducer, while generating the target string, have enough `memory' to ensure that the second adjunction for the tree-set is forced to occur? 

\end{document}

